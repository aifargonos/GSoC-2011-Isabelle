\documentclass{article}


\usepackage{geometry}
\geometry{a4paper}

\usepackage[parfill]{parskip}

\usepackage[utf8]{inputenc}
\usepackage[T1]{fontenc}

\usepackage{hyperref}



\title{GSoC -- Isabelle/jEdit document browser\\and various enhancements:\\Description and Design}
\author{Peter Skočovský (aifargonos)}
\date{}

\begin{document}
\maketitle



\section{The HTMLExport plugin}


\subsection{General description}

HTMLExport is a \href{http://www.jedit.org/}{jEdit} plugin working together
with \href{http://isabelle.in.tum.de/}{Isabelle/jEdit}. It was written for Isabelle/717880e98e6b
and has the same dependencies as Isabelle/jEdit.

HTMLExport plugin produces XHTML documents from Isabelle's \texttt{.thy} files
and displays the documents in its dockable window with \href{http://lobobrowser.org/cobra.jsp}{Cobra HTML Renderer}.
The document can be exported to a file specified by plugin's options.


\subsection{Conversion process}

The plugin traverses Isabelle commands obtained from \verb+snapshot.node.command_range()+
in function \verb+HTMLExportPlugin.convert(view: View)+.
The commands and their results are then decomposed by \verb+snapshot.state(command).markup+
and translated to XHTML. Markup commands are processed by local function \verb+process_text+
and their text arguments are interpreted by sub\LaTeX.
All other commands are processed by local function \verb+process_command+
and translated directly to XHTML as defined in function \verb+HTMLExportPlugin.xml_info_to_xhtml+.
Produced XHTML is then passed to \verb+Template+ where it is inserted intro a template of the document.



\section{sub\LaTeX}


\subsection{Design}

sub\LaTeX's basics were designed to be as similar to \LaTeX as possible.
For this purposes the parsing process was divided into character category management,
lexer, macro management and output layer.
Character category management assigns categories to characters as it is done in \TeX.
Lexer then groups these characters according to their categories into tokens.
Since syntactic layer of \TeX is not distinguishable from lexical layer,
lexer also handles context and interprets macros. Lexer outputs a sequence of output commands
that are interpreted by output layer and translated into XHTML.


\subsection{Implementation details}

\paragraph{Character category management} is implemented in \verb+CatCode+.
Instances of class \verb+CatCode+ represents \TeX character categories.
Method \verb+apply(c: Char): Boolean+ of these instances determines
whether a character belongs to a category represented by this instance.

\paragraph{Lexer} is implemented in \verb+Lexer+. It extends \verb+scala.util.parsing.combinator.Parsers+
and its \verb+type Elem+ is \verb+Char+.
Parser combinators \verb+is(cc: CatCode)+ and%\\
\verb+is_not(cc: CatCode)+ accepts characters according to their category.
All other parser combinators accept an result of previous parser as an argument
and returns parser that is meant to augment this argument with what it parses.
The parsing context is passed from parser to parser and altered during the parsing.
A type of this context is \verb+type Context+ and it is meant to store output queue and stacks for groups,
environments and macros. However, only stack for groups and output queue is implemented so far.
Parser combinator \verb+rep_arg+ chains parsers returned by \verb+pf+
such that result of previous parser is an argument for next call to \verb+pf+.

\paragraph{Grammar} (if it can be called grammar) is determined by these parser combinators:
\verb+body+ is alternative of \verb+syntactic_sugar+, \verb+command+, \verb+character+,
\verb+white_space+ and \verb+group+.

\verb+syntactic_sugar+ is implemented in \verb+Syntactic_Sugar+
for better modularity. It is meant to parse implicit paragraphs (two newlines), comments and
special sequences like \verb+``+. Problem is that implicit paragraphs must be implemented
in real preprocessing (simple replacement of two newlines by \verb+\par+)
in order for sub\LaTeX to work like \TeX.% So it will be best if comments are moved to Lexer.
Preprocessing can be implemented as a subclass of \verb+CharSequenceReader+
that does substitution at the beginning of source before returning rest.

\verb+command+ parses \LaTeX command and directly interprets it as a macro.
How parsing continues after command depends on what command it is so it is determined by macro management.

\verb+character+ and \verb+white_space+ parses simply one character and a sequence of white-space characters.

\verb+group+ parsers \LaTeX group and alters groups stack in context.
Context in the current implementation contains only one stack.
However it will be better if there are separate stacks for environments, for local macro definitions
and for output commands that have to be terminated at the end of the group (which is the only one implemented now).
Also it will be better if \verb+group+ was implemented by one function
that will, at its end, restore the groups stack as it was passed to it.

\paragraph{Macro management} is implemented in \verb+Macro+.
Each instance of \verb+class Macro+ specifies how is parsed what follows after its command string in method.
This is done by its method \verb+parser+ that returns parser which is meant to alter the context.
\verb+object Macro+ contains storage for macros and functions for its manipulation.
This storage will have to be connected with a macro stack in context in order to enable local macro definitions.
User definitions of macros are not possible yet, all implemented macros are defined in \verb+object Macro+.

Macro \verb+\normalfont+ cannot be implemented now. The \verb+group+ parser combinator have to be changed
as described above to enable implementation of this macro.

\paragraph{Output layer} is implemented in \verb+SubLaTeX+. It translates output commands obtained from
lexer's context into XHTML tags. The output commands are specified in \verb+Token+ now.
\verb+Token+ can be renamed into something more appropriate.
Upon creation of \verb+SubLaTeX+ objects, the lexer is initialized (now just counters are reset)
and the parsing is run by calling method \verb+apply+ of this object.

\paragraph{Counters} are implemented in \verb+Counter+ according to \LaTeX's counters.
However, there is no macro API yet.



\end{document}
